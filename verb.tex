\part{Глагол (the Verb)}
\section{Comparisons}
\sf
We are speaking different words and can't understand each other, but the true 
is that our languages are very similar. Many words have one common ancestor and 
they change not by accident, they changed according to some phonetical rule. 
One of them is transition of $p \rightarrow b \rightarrow v$.

\vspace{1cm}

%Таблица для греческого%
\begin{tabular}{| l | p{4cm} | p{4cm}|}
	\hline
	\multicolumn{3}{| c |}{Singularis}\\
	\hline
	1 & \greek{paideus\hlight{w}}    & я воспита\hlight{ю} \\ \hline
	2 & \greek{paideus\hlight{eis}}  & ты воспита\hlight{ешь} \\ \hline
	3 & \greek{1}{*}{paideuei}       & он воспитает \\ \hline
	\multicolumn{3}{| c |}{Pluralis} \\ \hline
	1 & \greek{paideusom\hlight{en}} & мы воспита\hlight{ем} \\ \hline
	2 & \greek{paideuse\hlight{te}}  & вы воспитае\hlight{те} \\ \hline
	3 & \greek{paideusousin}         & они воспитают \\ 
	\hline
\end{tabular}
%Здесь другая таблица для латыни%

\vspace{1cm}

\begin{tabular}{| l | p{4cm} | p{4cm}| l | l | l | l |} 
	\cline{1-3}
	\cline{5-7}
	\multicolumn{3}{| c |}{Singularis} & \, & \multicolumn{3}{| c |}{Imperativus}\\
	\cline{1-3}
	\cline{5-7}
	1 & orn\hlight{o} & я украша\hlight{ю}  & \, & ты & orn\hlight{a}
	  & украш\hlight{ай} \\
	\cline{5-7}
	\cline{1-3}
	2 & orna\hlight{s} & ты украша\hlight{ешь}  & \, & вы & orna\hlight{te}
	  & украшай\hlight{те} \\ 
	\cline{5-7} 
	\cline{1-3}
	3 & orna\hlight{t} & он украшае\hlight{т} \\ 
	\cline{1-3}
	%Pluralis%
	\multicolumn{3}{| c |}{Pluralis} \\
	\cline{1-3}
	1 & orna\hlight{mus} & мы украша\hlight{ем} \\
	\cline{1-3}
	2 & orna\hlight{tis} & вы украшае\hlight{те} \\
	\cline{1-3}
	3 & ornan\hlight{t} & они украшаю\hlight{т}\\ 
	\cline{1-3}
\end{tabular}

\vspace{1cm}

%Сравнение с португальским%
%Singularis%
\begin{tabular}{| l | p{4cm} | p{4cm}| l | l | l | l |} 
	\cline{1-3}
	\multicolumn{3}{| c |}{Singularis}\\
	\cline{1-3}
	1 & eu lav\hlight{o} & я мо\hlight{ю}\\
	\cline{1-3}
	2 & tu lava\hlight{s} & ты мое\hlight{шь}\\
	\cline{1-3}
	3 & ele lava & он моет \\ 
	\cline{1-3}
	%Pluralis%
	\multicolumn{3}{| c |}{Pluralis}\\
	\cline{1-3}
	1 & nós lava\hlight{mos} & мы мо\hlight{ем}\\
	\cline{1-3}
	2 & vós lava\hlight{is} & вы мое\hlight{те}\\
	\cline{1-3}
	3 & eles lavam & они моют\\ 
	\cline{1-3}
\end{tabular}
